%xelatex -shell-escape -output-directory=bin ergasia.tex
\documentclass{assignment}

\usepackage{enumerate} % Για την χρησιμοποίηση roman enumerate
\usepackage{paralist} % για το περιβάλλον inparaenum που είναι οι λίστες μέσα στο κείμενο.

\title{Αναγνώριση Προτύπων \\ Θέμα Εξαμήνου }
\date{Αθήνα, 2014}

\author{Αναγνωστόπουλος Βασίλης - Θάνος (ΜΠΠΛ 13002)}

\begin{document}

\maketitle
% Να σκεφτώ τί αλλαγές θέλω να κάνω με τις αριθμήσεις και άμα θέλω να κάνω.
% Να σκεφτώ να τις ενσωματώσω και στο assignment.cls

\setcounter{page}{1} 
\pagenumbering{roman}

\pagestyle{plain}
\tableofcontents
%\listoftables
\listoffigures
%\renewcommand\listoflistingscaption{Κατάλογος πηγαίου κώδικα}
%\listoflistings
\newpage

%\pagestyle{headings}
%\pagestyle{fancy}
\setcounter{page}{1} 
\pagenumbering{arabic}

\section{Άσκηση 1η}
\subsection{Εκφώνηση}

Να γίνει πλήρης βιβλιογραφική έρευνα με βάση τις λέξεις κλειδιά "\en{Artificial Immune System}".

\subsection {Λύση}

\section{Άσκηση 2η}
\subsection{Εκφώνηση}

Να γίνει πλήρης βιβλιογραφική έρευνα με βάση τις λέξεις κλειδιά "\en{Swarm Intelligence}".

\subsection {Λύση}

\section{Άσκηση 3η}
\subsection{Εκφώνηση}

Να γίνει πλήρης βιβλιογραφική έρευνα με βάση τις λέξεις κλειδιά "\en{Evolutionary Computing}" και \en{Genetic Algorithms}.

\subsection {Λύση}

σελ. 80

\section{Άσκηση 4η}
\subsection{Εκφώνηση}

Να γίνει πλήρης βιβλιογραφική έρευνα με βάση τις λέξεις κλειδιά "\en{Fuzzy Logic Systems}".

\subsection {Λύση}

σελ. 59

\section{Άσκηση 5η}
\subsection{Εκφώνηση}

Να υλοποιηθούν αλγόριθμοι ιεραρχικής ομαδοποίησης δεδομένων και να εφαρμοστούν στα δεδομένα εκφράσεων προσώπου που θα σας παρασχεθούν.

\subsection {Λύση}


σελ. 56



\phantomsection \label{Βιβλιογραφία}
\addcontentsline{toc}{section}{Βιβλιογραφία}
%\mtcaddchapter[Βιβλιογραφία] % Λόγω του minitoc
\bibliographystyle{plain}
\bibliography{references}

\newpage

\end{document}

