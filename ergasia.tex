%xelatex -shell-escape -output-directory=bin ergasia.tex
\documentclass{assignment}

\usepackage{enumerate} % Για την χρησιμοποίηση roman enumerate
\usepackage{paralist} % για το περιβάλλον inparaenum που είναι οι λίστες μέσα στο κείμενο.
\usepackage{program}
\usepackage{algorithm}

\title{Αναγνώριση Προτύπων \\ Θέμα Εξαμήνου }
\date{Αθήνα, 2014}

\author{Αναγνωστόπουλος Βασίλης - Θάνος (ΜΠΠΛ 13002) \and Βελισσαρίου Κυριάκος (ΜΠΠΛ 13005)}

\begin{document}

\maketitle
% Να σκεφτώ τί αλλαγές θέλω να κάνω με τις αριθμήσεις και άμα θέλω να κάνω.
% Να σκεφτώ να τις ενσωματώσω και στο assignment.cls

\setcounter{page}{1} 
\pagenumbering{roman}

\pagestyle{plain}
\tableofcontents
%\listoftables
\listoffigures

\makeatletter
\newcommand{\newalgname}[1]{%
  \renewcommand{\ALG@name}{#1}%
}
\newalgname{Αλγόριθμος}
\renewcommand{\listalgorithmname}{Κατάλογος αλγορίθμων}
\makeatother

\listofalgorithms

\renewcommand\listoflistingscaption{Κατάλογος πηγαίου κώδικα}
\renewcommand\listingscaption{Πηγαίος κώδικας}

\listoflistings
\newpage

%\pagestyle{headings}
%\pagestyle{fancy}
\setcounter{page}{1} 
\pagenumbering{arabic}

\section{Άσκηση 1η - \en{Artificial Immune System}}
\subsection{Εκφώνηση}

Να γίνει πλήρης βιβλιογραφική έρευνα με βάση τις λέξεις κλειδιά "\en{Artificial Immune System}".

\subsection {Λύση}

Το φυσικό ανοσοποιητικό σύστημα είναι ένα πολύπλοκο σύστημα για την λειτουργία του οποίου έχουν γραφτεί αρκετές δημοσιεύσεις για το πως λειτουργεί \cite{wiki:immune_system}. Εκτός από την ικανότητα του να καταπολεμά ξένα κύτταρα και μολύνσεις προς τον οργανισμό, διαθέτει μνήμη, μία ιδιαίτερα σημαντική ιδιότητα, καθ` ότι του επιτρέπει να αναγνωρίζει και να αντιμετωπίζει πιο άμεσα σε μία εισβολή από παθογόνα, που έχουν προσβάλει και παλιότερα τον οργανισμό. 

Ένα τεχνητό ανοσοποιητικό σύστημα (ΤΑΣ - αγγλ. \en{Artificial Immune System}) μοντελοποιεί την ικανότητα του φυσικού ανοσοποιητικού συστήματος των σπονδυλωτών να ανιχνεύει κύτταρα ξένα προς τον οργανισμό.Το αποτέλεσμα είναι ένα νέο υπολογιστικό μοντέλο το οποίο είναι ικανό να αναγνωρίζει πρότυπα και εφαρμόζεται κυρίως στην ανίχνευση ανωμαλιών \cite{engelbrecht,wiki:artificial_immune_system}.

Ο ορισμός και η ανάπτυξη ενός πλήρους ΤΑΣ περιλαμβάνει, γενικά, μία πληθώρα θεμάτων, μεταξύ των οποίων είναι \cite{engelbrecht,karakasis_thesis}:

\begin{itemize}
\item υβριδικές δομές και αλγόριθμοι, οι οποίοι λαμβάνουν υπ` υπόψιν τους μηχανισμούς του ανοσοποιητικού συστήματος όπως την ανίχνευση ξένων προτύπων με μία ορισμένη συγγένεια και την αποθήκευση πληροφορίας και την επαναχρησιμοποίηση της.
\item υπολογιστικοί αλγόριθμοι βασισμένοι σε αρχές του ανοσοποιητικού συστήματος, όπως είναι η κατανεμημένη επεξεργασία, η αρχή της επιλογής των κλώνων και η θεωρία του ανοσοποιητικού δικτύου.
\end{itemize}


Χρησιμοποιώντας τα παραπάνω ο \citeauthor{engelbrecht} παραθέτει τον βασικό αλγόριθμο για την δημιουργία ΤΑΣ (βλέπε αλγόριθμο \ref{algorith:AIS}). 
\begin{algorithm}                        % enter the algorithm environment
\caption{Βασικός αλγόριθμος ΤΑΣ \cite{engelbrecht}}          % give the algorithm a caption
\label{algorith:AIS}                      % and a label for \ref{} commands later in the document
\begin{program}
\mbox{Αρχικοποίηση ενός σύνολου τεχνικών λεμφοκυττάρων (ΤΛ) ως πληθυσμός }C;
\mbox{Καθορισμός των προτύπων των αντιγόνων ως σύνολο εκπαίδευσης }D_T;
\WHILE \mbox{κάποια συνθήκη τερματισμού είναι αναληθείς} \DO 
  \FOR \mbox{κάθε πρότυπο αντιγόνου } z_p \in D_T \DO
    \mbox{Επιλογή ενός υποσυνόλου ΤΛ για έκθεση στο }z_p \mbox{,σαν πληθυσμός } S \leq C;
    \FOR \mbox{για κάθε ΤΛ } x_i \in S \DO
          \mbox{Υπολογισμός την ομοιότητα του αντιγόνου μεταξύ } z_p,x_i ;
    \END
    \mbox{Επιλογή ενός υποσυνόλου ΤΛ που έχουν την μεγαλύτερη ομοιότητα }
    \mbox{αντιγόνων σαν πληθυσμός } H \leq S;
    \mbox{Προσαρμογή των ΤΛ με κάποια μέθοδο επιλογής, με βάση την υπολογισμένη}
    \mbox{ομοιότητα και/ή την ομοιότητα του δικτύου των ΤΛ στο }H;
    \mbox{Ανανέωση του βαθμού ομοιότητας των ΤΛ στο }H;
  \END
\END
\end{program}
\end{algorithm}

Κάθε κομμάτι του αλγορίθμου αναλύεται πιο κάτω \cite{engelbrecht}:
\begin{description}
\item[Αρχικοποίηση $C$ και καθορισμός $D_T$:] Ο πληθυσμός $C$ μπορεί να είναι είτε δημιουργημένος από τυχαία δημιουργημένα τεχνητά λεμφοκυττάρων \footnote{Το λεμφοκύτταρο αποτελεί είδος λευκού αιμοσφαιρίου το οποίο το συναντιόνται στο φυσικό ανοσοποιητικό σύστημα και είναι επιφορτισμένα με την άμυνα του οργανισμού έναντι σε λοιμώξεις \cite{wiki:lymphocytes}.} (ΤΛ) ή από κάποια άλλη μέθοδο η οποία εξαρτάται από τον αγλόριθμο του ΤΑΣ.

\item[Συνθήκη τερματισμού του while:] Στα περισσότερα μοντέλα των ΤΑΣ, η συνθήκη τερματισμού βασίζεται στην σύγκλιση του πληθυσμού των ΤΛ ή από ένα συγκεκριμένο αριθμό επαναλήψεων.

\item[Επιλογή του υποσυνόλου $S$ των ΤΛ:] Το υποσύνολο $S$ μπορεί να είναι ολόκληρο το σύνολο $P$ ή ένα τυχαίος αριθμός ΤΛ από το $P$. 

\item[Υπολογισμός της ομοιότητας του αντιγόνου:] Η ομοιότητα του αντιγόνου (αγγλ. \en{antigen affinity}) είναι η μέτρηση της "συγγένειας" που υπάρχει μεταξύ των ΤΛ και των προτύπων των αντιγόνων. 

\item[Επιλογή του υποσυνόλου $H$ των ΤΛ:] Σε κάποια από τα μοντέλα των ΤΑΣ, η επιλογή της καλύτερης ομοιότητας των ΤΛ βασίζεται σε κάποιο κατώφλι ομοιότητας. Έτσι το υποσύνολο $H$ μπορεί να είναι ολόκληρο το $S$, αναλόγως ποιο είναι το κατώφλι ομοιότητας.

\item[Υπολογισμός της ομοιότητας του δικτύου:] Αυτή είναι η μέτρηση της ομοιότητας μεταξύ δύο ΤΛ. Τα διάφορα μέτρα ομοιότητας ενός δικτύου είναι τα ίδια με εκείνα των αντιγόνων.. Ένα καθορισμένο κατώφλι ομοιότητας προσδιορίζει αν δύο ή περισσότερα ΤΛ συνδέονται για να σχηματίσουν ένα δίκτυο.

\item[Ανανέωση του βαθμού ομοιότητας των ΤΛ στο $H$:] Είναι η διαδικασία με την οποία τα ΤΛ ωριμάζουν. Η διαδικασία ωρίμανσης αλλάζει ανάλογα με το μοντέλο του ΤΑΣ.

Έχουν προταθεί αρκετοί αλγόριθμοι για την επίλυση των ΤΑΣ

\end{description}

\section{Άσκηση 2η}
\subsection{Εκφώνηση}

Να γίνει πλήρης βιβλιογραφική έρευνα με βάση τις λέξεις κλειδιά "\en{Swarm Intelligence}".

\subsection {Λύση}

\section{Άσκηση 3η}
\subsection{Εκφώνηση}

Να γίνει πλήρης βιβλιογραφική έρευνα με βάση τις λέξεις κλειδιά "\en{Evolutionary Computing}" και \en{Genetic Algorithms}.

\subsection {Λύση}

σελ. 80

\section{Άσκηση 4η - \en{Fuzzy Logic Systems}}
\subsection{Εκφώνηση}

Να γίνει πλήρης βιβλιογραφική έρευνα με βάση τις λέξεις κλειδιά "\en{Fuzzy Logic Systems}".

\subsection {Ασαφής Λογική}

Η ασαφής λογική πρόκειται για μία γενίκευση της συμβατικής Θεωρίας Συνόλων και είναι μία πλειότιμη λογική η οποία ασχολείται με την λογική σαν μία προσέγγιση και όχι σαν κάτι σταθερό και ακριβές \cite{engelbrecht,class_notes,wiki:fuzzy_logic,zadeh1994}. %Fuzzy logic is a form of many-valued logic which deals with reasoning that is approximate rather than fixed and exact.

Η ασαφής λογική βασίζεται στα ασαφή σύνολα. Σε σύγκριση με τα δυαδικά σύνολα (οι μεταβλητές των οποίων μπορούν να λάβουν την τιμή "αλήθεια" ή "ψευδές"), τα ασαφή σύνολα περιέχουν αντικείμενα τα οποία ικανοποιούν ανακριβείς ιδιότητες και μπορεί να έχουν μία τιμή αληθείας η οποία κυμαίνεται στο βαθμό μεταξύ 0 και 1 σε αυτό το σύνολο (βλ. εξίσωση \eqref{eq:fuzzy_set}) \cite{zadeh1965338}. 

\begin{equation}
\mu_A : X \rightarrow [0,1]
\label{eq:fuzzy_set}
\end{equation}

όπου $X$ ο χώρος των αντικειμένων. Αν $x$ ένα γενικό αντικείμενο του $X$ με $x \in X$ τότε το $\mu_A(x)$ δείχνει την βεβαιότητα με την οποία το στοιχείο $x$ ανήκει στο ασαφή σύνολο $A$.

Ο βαθμός συμμετοχής σε ένα ασαφές σύνολο υποδηλώνει την βεβαιότητα (ή την αβεβαιότητα) ότι το στοιχείο ανήκει σε αυτή την ομάδα. Η ασαφής λογική χρησιμοποιείται για να χειριστεί την έννοια της μερικής αλήθειας, όπου η τιμή της αλήθειας μπορεί να κυμαίνεται μεταξύ εντελώς αλήθειας και εντελώς ψευδή. Τα ασαφή σύνολα επιτρέπουν την μοντελοποίηση των αβεβαιοτήτων της φυσικής γλώσσας και μπορούν να χρησιμοποιηθούν τόσο για διακριτούς ή/και συνεχόμενους χώρους \cite{engelbrecht,class_notes,wiki:fuzzy_logic}. 

Για τον πλήρη ορισμό ενός ασαφές συνόλου χρειαζόμαστε ακόμα και την συνάρτηση μέλος του ασαφούς συνόλου, η οποία χρησιμοποιείται για να συνδέσει τον βαθμό συμμετοχής κάθε στοιχείου $x$ του χώρου των αντικειμένων με το αντίστοιχο ασαφές σύνολο. Οι συναρτήσεις μέλη μπορεί να έχουν οποιοδήποτε σχήμα ή τύπο (βλ. σχήμα \ref{fig:fuzzy_logic_membership_function}). \cite{engelbrecht}.

\begin{figure}[htbp]
  \centering
  \begin{subfigure}[b]{0.5\textwidth}
     \includegraphics[width=\textwidth,height=0.25\textheight]{images/fuzzy_logic_membership_function_triangular.png}
  \caption{Τριγωνική συνάρτηση μέλους}
  \end{subfigure}%
   ~ %add desired spacing between images, e. g. ~, \quad, \qquad, \hfill etc.
  \begin{subfigure}[b]{0.5\textwidth}
    \includegraphics[width=\textwidth,height=0.25\textheight]{images/fuzzy_logic_membership_function_trapezoidal.png}
  \caption{Τραπεζοειδής συνάρτηση μέλους}
  \end{subfigure}
   ~ %add desired spacing between images, e. g. ~, \quad, \qquad, \hfill etc.
  \begin{subfigure}[b]{0.4\textwidth}
    \includegraphics[width=\textwidth,height=0.25\textheight]{images/fuzzy_logic_membership_function_logistic.png}
  \caption{Λογιστική συνάρτηση μέλους}
  \end{subfigure}
   ~ %add desired spacing between images, e. g. ~, \quad, \qquad, \hfill etc.
  \begin{subfigure}[b]{0.5\textwidth}
    \includegraphics[width=\textwidth,height=0.25\textheight]{images/fuzzy_logic_membership_function_gaussian.png}
  \caption{Γκαουσιανή συνάρτηση μέλους}
  \end{subfigure}
  \caption{Παραδείγματα συναρτήσεων μελών \cite{engelbrecht}}
\label{fig:fuzzy_logic_membership_function}
\end{figure}

Τέλος στην ασαφή λογική μπορούν να χρησιμοποιηθούν και λεκτικές μεταβλητές (αγγλ. \en{linguistic variables}). Με τον όρο λεκτική μεταβλητή εννοούμε τις μεταβλητές των οποίων οι τιμές είναι λέξεις σε μία φυσική ή τεχνική γλώσσα \cite{zadeh1975concept}. Οι λεκτικές μεταβλητές επιτρέπουν την μετατροπή της φυσικής γλώσσας σε λογικές ή αριθμητικές παραστάσεις, οι οποίες παρέχουν τα εργαλεία για την προσεγγιστική λογική \cite{engelbrecht}.

\subsection{Συστήματα ασαφούς λογικής}

Η σχεδίαση συστημάτων ασαφούς λογικής (ΣΑΛ) είναι ένας από τους μεγαλύτερους τομείς εφαρμογής της ασαφούς λογικής \cite{engelbrecht}. Ένα σύστημα ασαφούς λογικής (αγγλ. \en{Fuzzy Logic Systems - FLS}) είναι μία γραμμική απεικόνιση ενός διανύσματος δεδομένων εισόδου (χαρακτηριστικά) σε μία βαθμωτή έξοδο (η οποία αποσυντίθεται σε μία συλλογή από ανεξάρτητες πολλαπλές εισόδου/μονής εξόδου σύστημα - βλέπε σχήμα \ref{fig:fuzzy_logic_system}) \cite{mendel364485,class_notes}.

\begin{figure}
\begin{center}
\resizebox*{\textwidth}{!}{
\includegraphics{images/fuzzy_logic_system.png}}
\caption{Σύστημα ασαφούς λογικής \cite{mendel364485,class_notes}}
\label{fig:fuzzy_logic_system}
\end{center}
\end{figure}

Στα ΣΑΛ, η δυναμική συμπεριφορά του συστήματος περιγράφεται από ένα σύνολο λεκτικών ασαφών κανόνων της μορφής ΑΝ - ΤΟΤΕ \cite{engelbrecht, class_notes}. Σε συνδυασμό, τα ασαφή σύνολα και οι λεκτικοί κανόνες αποτελούν την γνωσιακή βάση ενός ΣΑΛ. Ακόμα ένα ΣΑΛ αποτελείται από άλλα 3 
στοιχεία:

\begin{description}
\item[ασαφοποιητής (αγγλ. \en{fuzzifier}): ] ο σκοπός του ασαφοποιητή είναι η εύρεση μίας ασαφούς τιμής από μη ασαφές τιμές εισόδου. Αυτό επιτυχάνεται χρησιμοποιώντας την συνάρτηση μέλους που σχετίζεται με κάθε ασαφές σύνολο στα δεδομένα εισόδου. Δηλαδή, οι τιμές στα δεδομένα εισόδου αντιστοιχούνται σε βαθμούς συμμετοχής σε ασαφή σύνολα \cite{engelbrecht}.

\item [συμπερασματοποιητής (αγγλ. \en{inference}): ] o σκοπός του συμπερασματοποιητή (αγγλ. \en{inference}) είναι να χαρτογραφήσει τις ασαφές εισόδους (όπως λαμβάνονται από την διαδικασία της ασαφοποίησης (αγγλ. \en{fuzzifier}) στους λεκτικούς κανόνες και να παράγουν μία αποσαφηνισμένη (αγγλ. \en{fuzzified}) έξοδο για κάθε κανόνα \cite{engelbrecht}.

\item [αποσαφηνιστής(αγγλ. \en{defuzzifier}): ] ο σκοπός του αποσαφηνιστή (αγγλ. \en{defuzzifier}) είναι η παραγωγή μίας σαφούς εξόδους για το ΣΑΛ από το ασαφές σύνολο που προέκυψε ως έξοδος από τον συμπερασματοποιητή. Δηλαδή παραγάγει σε αυτό το στάδιο γίνεται η μετατροπή των ασαφών κανόνων σε μία βαθμωτή τιμή ή γενικά σε μή ασαφή μεταβλητή \cite{class_notes,engelbrecht}. Στην βιβλιογραφία έχουν προταθεί αρκετοί αποσαφηνιστές για την εύρεση της βαθμωτής τιμής που αναπαριστά την ενέργεια που πρέπει να πραγματοποιηθεί.
\end{description}

Καθένα από αυτά τα στοιχεία εκτελεί μία συγκεκριμένη εργασία κατά την διαδικασία συλλογισμού (αγγλ. \en{reasoning process}). Τα διαφορετικά στοιχεία ενός ΣΑΛ φαίνονται στο σχήμα \ref{fig:fuzzy_logic_system_2}.

\begin{figure}
\begin{center}
\resizebox*{\textwidth}{!}{
\includegraphics{images/fuzzy_logic_system_2.png}}
\caption{Σύστημα ασαφούς λογικής (2) \cite{engelbrecht}}
\label{fig:fuzzy_logic_system_2}
\end{center}
\end{figure}


\section{Άσκηση 5η}
\subsection{Εκφώνηση}

Να υλοποιηθούν αλγόριθμοι ιεραρχικής ομαδοποίησης δεδομένων και να εφαρμοστούν στα δεδομένα εκφράσεων προσώπου που θα σας παρασχεθούν.

\subsection {Λύση}

Η ιεραρχική ομαδοποίηση (αγγλ. \en{hierarchical clustering}) είναι μία μέθοδος ανάλυσης κλάσεων η οποία αναζητά να κατασκευάσει μία ιεραρχία ομάδων. Η ιεραρχική ομαδοποίηση έχει την ιδιότητα ότι τα δείγματα που ανήκουν στην ίδια ομάδα σε κάποιο επίπεδο να παραμένουν στην ίδια ομάδα σε υψηλότερα επίπεδα \cite{wiki:hierarchical_clustering, class_notes}.

Σε κάθε ιεραρχική ομαδοποίηση αντιστοιχεί ένα δενδρόγραμμα (αγγλ. \en{dendogram}).

  %\begin{tabular}{|c|c|m{0.35\textwidth}|m{0.35\textwidth}|m{2.0cm}|c|m{1.5cm}|}

\begin{table}[htbp]
\begin{center}
  \begin{tabular}{|m{0.35\textwidth}|m{0.6\textwidth}|}
    \hline
    {\bf Όνομα} & {\bf Τύπος}  \\ \hline
    Ευκλείδια απόσταση   & $\|a-b \|_2 = \sqrt{\sum_i (a_i-b_i)^2}$  \\ \hline
    Απόσταση manhattan   & $\|a-b \|_1 = \sum_i \|a_i-b_i \|$ \\ \hline
    Μέγιστη απόσταση     & $\|a-b \|_\infty = \max_i \|a_i-b_i\|$ \\ \hline
    Απόσταση mahalanobis & $ \sqrt{(a-b)^{\top}S^{-1}(a-b)} $, όπου $S$ είνια ο πίνακας συμεταβλητότητας   \\ \hline
  \end{tabular}
\caption{Ο πίνακας με τις αποστάσεις \cite{wiki:hierarchical_clustering}.}
\label{table:distances}
\end{center}
\end{table}


\begin{table}[htbp]
\begin{center}
  \begin{tabular}{|m{0.35\textwidth}|m{0.6\textwidth}|}
    \hline
    {\bf Όνομα} & {\bf Τύπος}  \\ \hline
    Πλήρης σύνδεσης   & $\max \, \{\, d(a,b) : a \in A,\, b \in B \,\}$  \\ \hline
    Μονή σύνδεση   & $ \min \, \{\, d(a,b) : a \in A,\, b \in B \,\}$ \\ \hline
    Μέση σύνδεση   & $\frac{1}{|A| |B|} \sum_{a \in A }\sum_{ b \in B} d(a,b)$ \\ \hline
  \end{tabular}
\caption{Ο πίνακας με τα κριτήρια σύνδεσης \cite{wiki:hierarchical_clustering}.}
\label{table:linkage}
\end{center}
\end{table}


\captionof{listing}{Ο κώδικας του \en{matlab}}
\inputminted[breaklines=true, frame=lines, framesep=2mm, baselinestretch=1.2, fontsize=\footnotesize, linenos]{matlab}{../matlab/features.m} 

\begin{figure}[htbp]
  \centering
  \begin{subfigure}[b]{0.5\textwidth}
     \includegraphics[width=\textwidth,height=0.25\textheight]{matlab/labeled_data_points.png}
  \caption{Τα ταξινομημένα δεδομένα}
  \end{subfigure}%
   ~ %add desired spacing between images, e. g. ~, \quad, \qquad, \hfill etc.
  \begin{subfigure}[b]{0.5\textwidth}
    \includegraphics[width=\textwidth,height=0.25\textheight]{matlab/unlabeled_data_points.png}
  \caption{Τα αταξινομημένα δεδομένα}
  \end{subfigure}

  \caption{Τα δεδομένα εκφράσεων προσώπου}
\label{fig:data_points}
\end{figure}

\begin{figure}[htbp]
  \centering
  \begin{subfigure}[b]{0.5\textwidth}
     \includegraphics[width=\textwidth,height=0.25\textheight]{matlab/hierarchical_dendogram_average_euclidean.png}
  \caption{Τα δενδρογραμμα των δεδομένα}
  \end{subfigure}%
   ~ %add desired spacing between images, e. g. ~, \quad, \qquad, \hfill etc.
  \begin{subfigure}[b]{0.5\textwidth}
    \includegraphics[width=\textwidth,height=0.25\textheight]{matlab/identified_clusters_average_euclidean.png}
  \caption{Οι συστάδες που δημιουργήθηκαν}
  \end{subfigure}

  \caption{Η ιεραρχική ομαδοποίηση των δεδομένων χρησιμοποιώντας την μέση σύνδεση και την ευκλείδεια απόσταση}
\label{fig:clustering_average_euclidean}
\end{figure}

\begin{figure}[htbp]
  \centering
  \begin{subfigure}[b]{0.5\textwidth}
     \includegraphics[width=\textwidth,height=0.25\textheight]{matlab/hierarchical_dendogram_average_mahalanobis.png}
  \caption{Τα δενδρογραμμα των δεδομένα}
  \end{subfigure}%
   ~ %add desired spacing between images, e. g. ~, \quad, \qquad, \hfill etc.
  \begin{subfigure}[b]{0.5\textwidth}
    \includegraphics[width=\textwidth,height=0.25\textheight]{matlab/identified_clusters_average_mahalanobis.png}
  \caption{Οι συστάδες που δημιουργήθηκαν}
  \end{subfigure}

  \caption{Η ιεραρχική ομαδοποίηση των δεδομένων χρησιμοποιώντας την μέση σύνδεση και την απόσταση mahalanobis}
\label{fig:clustering_average_mahalanobis}
\end{figure}

\begin{figure}[htbp]
  \centering
  \begin{subfigure}[b]{0.5\textwidth}
     \includegraphics[width=\textwidth,height=0.25\textheight]{matlab/hierarchical_dendogram_average_manhattan.png}
  \caption{Τα δενδρογραμμα των δεδομένα}
  \end{subfigure}%
   ~ %add desired spacing between images, e. g. ~, \quad, \qquad, \hfill etc.
  \begin{subfigure}[b]{0.5\textwidth}
    \includegraphics[width=\textwidth,height=0.25\textheight]{matlab/identified_clusters_average_manhattan.png}
  \caption{Οι συστάδες που δημιουργήθηκαν}
  \end{subfigure}
  \caption{Η ιεραρχική ομαδοποίηση των δεδομένων χρησιμοποιώντας την μέση σύνδεση και την απόσταση manhattan}
\label{fig:clustering_average_manhattan}
\end{figure}

\begin{figure}[htbp]
  \centering
  \begin{subfigure}[b]{0.5\textwidth}
     \includegraphics[width=\textwidth,height=0.25\textheight]{matlab/hierarchical_dendogram_single_euclidean.png}
  \caption{Τα δενδρογραμμα των δεδομένα}
  \end{subfigure}%
   ~ %add desired spacing between images, e. g. ~, \quad, \qquad, \hfill etc.
  \begin{subfigure}[b]{0.5\textwidth}
    \includegraphics[width=\textwidth,height=0.25\textheight]{matlab/identified_clusters_single_euclidean.png}
  \caption{Οι συστάδες που δημιουργήθηκαν}
  \end{subfigure}
  \caption{Η ιεραρχική ομαδοποίηση των δεδομένων χρησιμοποιώντας την μονή σύνδεση και την ευκλείδεια απόσταση}
\label{fig:clustering_single_euclidean}
\end{figure}

\begin{figure}[htbp]
  \centering
  \begin{subfigure}[b]{0.5\textwidth}
     \includegraphics[width=\textwidth,height=0.25\textheight]{matlab/hierarchical_dendogram_complete_euclidean.png}
  \caption{Τα δενδρογραμμα των δεδομένα}
  \end{subfigure}%
   ~ %add desired spacing between images, e. g. ~, \quad, \qquad, \hfill etc.
  \begin{subfigure}[b]{0.5\textwidth}
    \includegraphics[width=\textwidth,height=0.25\textheight]{matlab/identified_clusters_complete_euclidean.png}
  \caption{Οι συστάδες που δημιουργήθηκαν}
  \end{subfigure}
  \caption{Η ιεραρχική ομαδοποίηση των δεδομένων χρησιμοποιώντας την πλήρη σύνδεση και την ευκλείδεια απόσταση}
\label{fig:clustering_complete_euclidean}
\end{figure}



\phantomsection \label{Βιβλιογραφία}
\addcontentsline{toc}{section}{Βιβλιογραφία}
%\mtcaddchapter[Βιβλιογραφία] % Λόγω του minitoc
\bibliographystyle{plainnat}
\bibliography{references}

\newpage

\end{document}

